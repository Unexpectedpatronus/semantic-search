\nonumchapter{ОПРЕДЕЛЕНИЯ}

	
	В настоящей расчетно-пояснительной записке применяют следующие термины с соответствующими определениями:
	
	\textbf{Семантический поиск} – метод информационного поиска, основанный на понимании смыслового содержания поискового запроса и документов, а не только на сопоставлении ключевых слов.
	
	\textbf{Векторное представление документов} – способ представления текстовых документов в виде числовых векторов в многомерном пространстве, где семантически близкие документы располагаются рядом друг с другом.
	
	\textbf{Doc2Vec} – алгоритм машинного обучения, расширяющий модель Word2Vec для создания векторных представлений документов произвольной длины.
	
	\textbf{Корпус документов} – структурированная коллекция текстовых документов, используемая для обучения модели машинного обучения.
	
	\textbf{Токенизация} – процесс разбиения текста на отдельные элементы (токены), такие как слова, знаки препинания или другие значимые единицы.
	
	\textbf{Лемматизация} – процесс приведения слова к его словарной форме (лемме) с учетом морфологического анализа.
	
	\textbf{Косинусное сходство} – мера сходства между двумя векторами, вычисляемая как косинус угла между ними в многомерном пространстве.
	
	\textbf{Экстрактивная суммаризация} – метод автоматического реферирования текста путем выделения наиболее важных предложений из исходного документа.
	
	\textbf{Distributed Memory (DM)} – режим обучения в алгоритме Doc2Vec, при котором модель учитывает контекст документа при предсказании слов.
	
	\textbf{Distributed Bag of Words (DBOW)} – режим обучения в алгоритме Doc2Vec, при котором модель предсказывает слова документа без учета их порядка.
	
\clearpage