\nonumchapter{ВВЕДЕНИЕ}

	
	В современном информационном обществе объем текстовых данных растет экспоненциально. Корпоративные архивы, научные библиотеки, юридические базы данных содержат миллионы документов, и эффективный поиск необходимой информации становится критически важной задачей. Традиционные методы поиска, основанные на сопоставлении ключевых слов, часто не обеспечивают требуемого качества результатов, так как не учитывают семантические связи между понятиями, синонимы и контекст использования терминов.
	
	Актуальность темы исследования обусловлена необходимостью разработки интеллектуальных систем поиска, способных понимать смысловое содержание документов и запросов пользователей. Применение технологий машинного обучения, в частности алгоритмов векторного представления текстов, открывает новые возможности для создания систем семантического поиска, превосходящих по эффективности классические подходы.
	
	Объектом исследования является процесс поиска информации в больших массивах текстовых документов различных форматов.
	
	Предметом исследования выступают методы и алгоритмы семантического анализа текстов на основе технологий машинного обучения, в частности алгоритм Doc2Vec.
	
	Целью дипломной работы является разработка программной системы семантического поиска по документам с использованием технологии машинного обучения Doc2Vec для повышения качества и релевантности результатов поиска.
	
	Для достижения поставленной цели необходимо решить следующие задачи:
	
	1. Провести анализ существующих методов информационного поиска и выявить их ограничения.
	
	2. Исследовать алгоритмы векторного представления текстов и обосновать выбор технологии Doc2Vec.
	
	3. Спроектировать архитектуру системы семантического поиска с учетом требований производительности и масштабируемости.
	
	4. Разработать модули обработки документов различных форматов (PDF, DOCX, DOC).
	
	5. Реализовать алгоритм обучения модели Doc2Vec на корпусе документов.
	
	6. Создать поисковый движок с поддержкой семантических запросов.
	
	7. Разработать модуль автоматической суммаризации документов.
	
	8. Реализовать графический и командный интерфейсы пользователя.
	
	9. Провести сравнительное исследование разработанной системы с классическими методами поиска (TF-IDF, BM25).
	
	10. Оценить экономическую эффективность предложенного решения.
	
	Методы исследования включают системный анализ, математическое моделирование, методы машинного обучения, экспериментальные исследования, сравнительный анализ.
	
	Научная новизна работы заключается в комплексном подходе к решению задачи семантического поиска, включающем адаптацию алгоритма Doc2Vec для работы с многоязычными документами, разработку оригинальных методов предобработки текстов и создание гибридного алгоритма суммаризации, сочетающего статистические и семантические подходы.
	
	Практическая значимость работы определяется возможностью применения разработанной системы в различных областях: корпоративном документообороте, научных исследованиях, юридической практике, медиа-аналитике. Система позволяет существенно сократить время поиска необходимой информации и повысить полноту выдачи релевантных документов.
	
	Структура работы. Расчетно-пояснительная записка состоит из введения, четырех основных разделов, заключения, списка использованных источников и приложений. В аналитическом разделе проводится исследование предметной области и существующих решений. Конструкторский раздел посвящен проектированию архитектуры системы. В технологическом разделе описывается реализация основных компонентов. Исследовательский раздел содержит результаты экспериментального сравнения с классическими методами поиска.
	
\clearpage