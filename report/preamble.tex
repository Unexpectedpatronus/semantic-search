% preamble.tex - упрощенная версия с улучшенными листингами
\pdfminorversion=7
% Класс документа с поддержкой 14pt
\documentclass[14pt,a4paper]{extreport}


% Настройка кодировки и языков
\usepackage{cmap}             % обеспечивает поиск и копирование
\usepackage[T2A]{fontenc}  	  % кириллица + латиница
\usepackage[utf8]{inputenc}   % кодировка исходника
\usepackage[russian,english]{babel}

% Геометрия страницы
\usepackage[left=3cm,right=1cm,top=2cm,bottom=2cm]{geometry}

% Интервалы и отступы
\usepackage{setspace}
\onehalfspacing
\usepackage{indentfirst}
\setlength{\parindent}{1.25cm}

% Колонтитулы
\usepackage{fancyhdr}
\pagestyle{fancy}
\fancyhf{}
\fancyfoot[C]{\thepage}
\renewcommand{\headrulewidth}{0pt}
\renewcommand{\footrulewidth}{0pt}
\setlength{\headheight}{15mm}
\setlength{\footskip}{20pt}

% Выравнивание текста
\usepackage{ragged2e}

% Переносы
\usepackage{hyphenat}
\sloppy

% Математика
\usepackage{amsmath}
\usepackage{amssymb}
\usepackage{amsthm}

% Таблицы
\usepackage{array}
\usepackage{tabularx}
\usepackage{booktabs}
\usepackage{multirow}
\usepackage{longtable}

% КРИТИЧЕСКИ ВАЖНЫЙ ПОРЯДОК ЗАГРУЗКИ ПАКЕТОВ ДЛЯ ГРАФИКИ
% 1. Сначала загружаем графику
\usepackage{graphicx}

% 2. Затем загружаем float БЕЗ caption
\usepackage{float}

% 3. И только в самом конце загружаем caption
\usepackage[font=singlespacing,justification=centering]{caption}

% Настройка подписей после загрузки caption
\DeclareCaptionLabelSeparator{mysep}{~---~}
\captionsetup{
	figurename=Рисунок,
	tablename=Таблица,
	labelsep=mysep
}

% Списки
\usepackage{enumitem}
\setlist{nolistsep}

% Настройка маркера для всех уровней itemize
\renewcommand{\labelitemi}{---}      % первый уровень
\renewcommand{\labelitemii}{---}     % второй уровень  
\renewcommand{\labelitemiii}{---}    % третий уровень
\renewcommand{\labelitemiv}{---}     % четвертый уровень

% ===== УЛУЧШЕННАЯ РАБОТА С ЛИСТИНГАМИ =====

% Пакеты для листингов
\usepackage{xcolor}
\usepackage{listings}
\usepackage{listingsutf8}

% Определение цветов для листингов (для черно-белой печати используем оттенки серого)
\definecolor{codegreen}{gray}{0.4}
\definecolor{codegray}{gray}{0.5}
\definecolor{codepurple}{gray}{0.3}

% Настройка выравнивания заголовков листингов по левому краю
\captionsetup[lstlisting]{
	justification=raggedright,
	singlelinecheck=false,
	font=singlespacing,
	labelsep=mysep
}

% Базовые настройки listings для черно-белой печати
\lstset{
	language=Python,
	basicstyle=\footnotesize\ttfamily,
	keywordstyle=\bfseries,
	commentstyle=\itshape,
	stringstyle=\ttfamily,
	showstringspaces=false,
	breaklines=true,
	breakatwhitespace=true,
	frame=single,
	numbers=left,
	numberstyle=\tiny,
	numbersep=5pt,
	captionpos=t,
	tabsize=4,
	keepspaces=true,
	columns=fullflexible,
	extendedchars=true,
	framerule=0.8pt,
	literate={а}{{\cyra}}1
	{б}{{\cyrb}}1
	{в}{{\cyrv}}1
	{г}{{\cyrg}}1
	{д}{{\cyrd}}1
	{е}{{\cyre}}1
	{ё}{{\cyryo}}1
	{ж}{{\cyrzh}}1
	{з}{{\cyrz}}1
	{и}{{\cyri}}1
	{й}{{\cyrishrt}}1
	{к}{{\cyrk}}1
	{л}{{\cyrl}}1
	{м}{{\cyrm}}1
	{н}{{\cyrn}}1
	{о}{{\cyro}}1
	{п}{{\cyrp}}1
	{р}{{\cyrr}}1
	{с}{{\cyrs}}1
	{т}{{\cyrt}}1
	{у}{{\cyru}}1
	{ф}{{\cyrf}}1
	{х}{{\cyrh}}1
	{ц}{{\cyrc}}1
	{ч}{{\cyrch}}1
	{ш}{{\cyrsh}}1
	{щ}{{\cyrshch}}1
	{ъ}{{\cyrhrdsn}}1
	{ы}{{\cyrery}}1
	{ь}{{\cyrsftsn}}1
	{э}{{\cyrerev}}1
	{ю}{{\cyryu}}1
	{я}{{\cyrya}}1
	{А}{{\CYRA}}1
	{Б}{{\CYRB}}1
	{В}{{\CYRV}}1
	{Г}{{\CYRG}}1
	{Д}{{\CYRD}}1
	{Е}{{\CYRE}}1
	{Ё}{{\CYRYO}}1
	{Ж}{{\CYRZH}}1
	{З}{{\CYRZ}}1
	{И}{{\CYRI}}1
	{Й}{{\CYRISHRT}}1
	{К}{{\CYRK}}1
	{Л}{{\CYRL}}1
	{М}{{\CYRM}}1
	{Н}{{\CYRN}}1
	{О}{{\CYRO}}1
	{П}{{\CYRP}}1
	{Р}{{\CYRR}}1
	{С}{{\CYRS}}1
	{Т}{{\CYRT}}1
	{У}{{\CYRU}}1
	{Ф}{{\CYRF}}1
	{Х}{{\CYRH}}1
	{Ц}{{\CYRC}}1
	{Ч}{{\CYRCH}}1
	{Ш}{{\CYRSH}}1
	{Щ}{{\CYRSHCH}}1
	{Ъ}{{\CYRHRDSN}}1
	{Ы}{{\CYRERY}}1
	{Ь}{{\CYRSFTSN}}1
	{Э}{{\CYREREV}}1
	{Ю}{{\CYRYU}}1
	{Я}{{\CYRYA}}1
	{_}{{\_}}1
	{__}{{\_\_}}2
}

% Команда для вставки листингов из файлов
\newcommand{\lstinputpath}[2][]{%
	\lstinputlisting[#1]{#2}%
}

% Окружение для листингов с ручной нумерацией (для разбиваемых листингов)
\lstnewenvironment{manuallisting}[2][]{%
	\captionsetup[lstlisting]{labelformat=empty}%
	\lstset{caption={#2},#1}%
}{%
	\captionsetup[lstlisting]{labelformat=default}%
}

% ===== КОНЕЦ НАСТРОЕК ЛИСТИНГОВ =====

% Для включения титульного PDF
\usepackage{pdfpages}

% Заголовки
\usepackage{titlesec}
\usepackage{tocloft}

% Нумерованные главы — номер по красной строке
\titleformat{\chapter}[block]
{\normalfont\bfseries\large\raggedright}
{\hspace*{1.25cm}\thechapter.}
{1em}
{\MakeUppercase}

% Нумерованные секции
\titleformat{\section}[block]
{\normalfont\bfseries\large\raggedright}
{\hspace*{1.25cm}\thesection}
{1em}
{}

% Нумерованные подсекции
\titleformat{\subsection}[block]
{\normalfont\bfseries\normalsize\raggedright}
{\hspace*{1.25cm}\thesubsection}
{1em}
{}

% Ненумерованные главы по центру
\titleformat{name=\chapter,numberless}[block]
{\centering\normalfont\bfseries\large}
{}
{0pt}
{\MakeUppercase}

% Настройка интервалов вокруг заголовков
\titlespacing*{\chapter}{0pt}{-50pt}{30pt}
\titlespacing*{\section}{0pt}{20pt}{10pt}
\titlespacing*{\subsection}{0pt}{15pt}{10pt}

% Настройка содержания
% Делаем заголовок содержания как ненумерованную главу
\renewcommand{\cfttoctitlefont}{\hfill\normalfont\bfseries\large\MakeUppercase}
\renewcommand{\cftaftertoctitle}{\hfill}
% Обычный шрифт для пунктов
\renewcommand{\cftchapfont}{\normalfont}
\renewcommand{\cftchappagefont}{\normalfont}
\renewcommand{\cftsecfont}{\normalfont}
\renewcommand{\cftsubsecfont}{\normalfont}
% Добавляем точки после номеров разделов
\renewcommand{\cftchapaftersnum}{.}
\renewcommand{\cftsecaftersnum}{.}
\renewcommand{\cftsubsecaftersnum}{.}
% Настройка отступов (как у глав)
\setlength{\cftbeforetoctitleskip}{-45pt} % Как у \titlespacing*{\chapter}
\setlength{\cftaftertoctitleskip}{30pt}   % Как у \titlespacing*{\chapter}
\setlength{\cftbeforechapskip}{0pt}

% Пакет для подчеркивания (необходим для титульного листа)
\usepackage[normalem]{ulem}

% Ссылки - ЗАГРУЖАЕМ В САМОМ КОНЦЕ
\usepackage[unicode,colorlinks=false,pdfborder={0 0 0}]{hyperref}

% Приложения  
\usepackage[titletoc]{appendix}

% Команды для ненумерованных разделов
\newcommand{\nonumchapter}[1]{%
	\chapter*{#1}%
	\addcontentsline{toc}{chapter}{#1}%
}

% Настройка формул
\renewcommand{\theequation}{\arabic{equation}}

% Настройка нумерации глав без ведущих нулей
\renewcommand{\thechapter}{\arabic{chapter}}
\renewcommand{\thesection}{\thechapter.\arabic{section}}
\renewcommand{\thesubsection}{\thesection.\arabic{subsection}}

% Настройка нумерации таблиц и рисунков
\counterwithin{figure}{chapter}
\counterwithin{table}{chapter}
\counterwithin{equation}{chapter}

% Команда для подписи таблиц
\newcommand{\tabcaption}[2]{
	\begin{center}
		Таблица #1~---~#2
	\end{center}
}

% Переопределение окружения для формул с пояснениями
\newenvironment{formulawhere}{%
	\vspace{\baselineskip}%
	\begin{equation}%
	}{%
	\end{equation}%
	\vspace{-\baselineskip}%
	\noindent где\vspace{-0.5\baselineskip}%
}

% Макрос для пояснения символов в формулах
\newcommand{\whereitem}[2]{%
	\par\noindent\hspace{1.25cm}#1~---~#2;%
}
