\nonumchapter{ЗАКЛЮЧЕНИЕ}

	В результате выполнения дипломной работы достигнута поставленная цель – разработана программная система семантического поиска по документам с использованием технологии машинного обучения Doc2Vec, обеспечивающая повышение качества и релевантности результатов поиска.
	
	В ходе работы решены все поставленные задачи:
	
	1. Проведен комплексный анализ существующих методов информационного поиска, выявлены их ограничения, связанные с отсутствием понимания семантики текста. Показано, что традиционные методы (TF-IDF, BM25) не обеспечивают требуемого качества при работе со сложными запросами.
	
	2. Исследованы современные алгоритмы векторного представления текстов. Обоснован выбор технологии Doc2Vec как оптимального решения, обеспечивающего баланс между качеством результатов и вычислительными требованиями.
	
	3. Спроектирована модульная архитектура системы, обеспечивающая масштабируемость, расширяемость и высокую производительность. Архитектура включает уровни представления, бизнес-логики и доступа к данным.
	
	4. Разработаны эффективные модули обработки документов различных форматов (PDF, DOCX, DOC) с оптимизацией для больших файлов. Реализована потоковая обработка и поддержка многоязычных документов.
	
	5. Реализован адаптивный алгоритм обучения модели Doc2Vec, учитывающий размер корпуса и языковой состав документов. Алгоритм автоматически адаптирует параметры для достижения оптимального качества.
	
	6. Создан высокопроизводительный поисковый движок с поддержкой семантических запросов, кэшированием результатов и возможностью фильтрации. Среднее время выполнения запроса составляет 23 мс.
	
	7. Разработан модуль автоматической суммаризации документов на основе алгоритма TextRank с семантическим ранжированием предложений. Модуль позволяет создавать информативные выжимки с коэффициентом сжатия до 80\%.
	
	8. Реализованы удобные интерфейсы пользователя: графический на базе PyQt6 и командной строки на базе Click. Интерфейсы обеспечивают доступ ко всем функциям системы.
	
	9. Проведено экспериментальное сравнение с классическими методами поиска. Доказано превосходство разработанной системы по метрике MAP на 34\% над BM25 и на 50\% над TF-IDF.
	
	10. Оценена экономическая эффективность решения. Показано, что система окупается менее чем за месяц по сравнению с облачными сервисами при регулярном использовании.
	
	\textbf{Основные результаты работы:}
	
	1. Разработана полнофункциональная система семантического поиска, готовая к практическому применению в различных областях.
	
	2. Достигнуто существенное повышение качества поиска по сравнению с традиционными методами, особенно для сложных семантических запросов.
	
	3. Обеспечена высокая производительность: обработка 10000 документов за 15 минут на 8-ядерном процессоре.
	
	4. Реализована поддержка многоязычных корпусов документов с автоматической адаптацией параметров обработки.
	
	5. Создана расширяемая архитектура, позволяющая добавлять новые форматы документов и методы анализа.
	
	\textbf{Научная новизна} работы заключается в:
	\begin{itemize}
		\item Разработке адаптивного алгоритма обучения Doc2Vec для многоязычных корпусов
		\item Создании гибридного метода суммаризации, сочетающего статистический и семантический подходы
		\item Реализация алгоритмов для работы с большими документами
	\end{itemize}
	
	\textbf{Практическая значимость} определяется возможностью применения системы в:
	\begin{itemize}
		\item Корпоративных системах управления документами
		\item Научных библиотеках и репозиториях
		\item Юридических информационных системах
		\item Медиа-архивах и издательствах
	\end{itemize}
	
	\textbf{Перспективы развития:}
	
	1. Интеграция с современными языковыми моделями (BERT, GPT) для повышения качества понимания контекста.
	
	2. Разработка веб-интерфейса для обеспечения удаленного доступа к системе.
	
	3. Реализация распределенной обработки для работы с корпусами миллионного масштаба.
	
	4. Добавление поддержки дополнительных языков и форматов документов.
	
	5. Создание API для интеграции с существующими корпоративными системами.
	
	Разработанная система семантического поиска представляет собой эффективное решение актуальной проблемы информационного поиска в больших массивах документов. Применение технологий машинного обучения позволило создать инструмент, существенно превосходящий традиционные подходы по качеству результатов при сохранении приемлемой производительности.
	