% referat.tex - Реферат (без \begin{document} и \end{document})
\nonumchapter{РЕФЕРАТ}

Расчетно-пояснительная записка включает 64 страниц, 13 рисунков, 7~таблиц, 18 источников, 3 приложения.

Ключевые слова: СЕМАНТИЧЕСКИЙ ПОИСК, МАШИННОЕ ОБУЧЕНИЕ, DOC2VEC, ОБРАБОТКА ЕСТЕСТВЕННОГО ЯЗЫКА, PYTHON, ИНФОРМАЦИОННЫЙ ПОИСК, ВЕКТОРНОЕ ПРЕДСТАВЛЕНИЕ ДОКУМЕНТОВ.

Объектом исследования является процесс поиска информации в больших массивах текстовых документов.

Цель работы -- разработка программной системы семантического поиска по документам с использованием технологии машинного обучения Doc2Vec для повышения качества и релевантности результатов поиска.

В процессе работы проводились исследования существующих методов информационного поиска, анализ алгоритмов векторного представления текстов, проектирование архитектуры системы, разработка программного обеспечения на языке Python с использованием библиотек Gensim, SpaCy, PyQt6.

В результате разработана полнофункциональная система семантического поиска, включающая модули обработки документов различных форматов (PDF, DOCX, DOC), обучения моделей Doc2Vec, выполнения поисковых запросов с учетом семантической близости, автоматической суммаризации документов. Система обеспечивает повышение качества поиска на 34\% по сравнению с классическими методами TF-IDF и BM25.

Степень внедрения -- опытная эксплуатация.

Область применения -- корпоративные системы управления документами, научные библиотеки, юридические информационные системы.

Экономическая эффективность работы обусловлена сокращением времени поиска необходимой информации на 70\% и отсутствием необходимости в платных API сторонних сервисов.