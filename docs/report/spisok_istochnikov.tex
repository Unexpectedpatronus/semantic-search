\nonumchapter{СПИСОК ИСПОЛЬЗОВАННЫХ ИСТОЧНИКОВ}

1. Агеев, М. С. Официальные метрики РОМИП'2010 / М. С. Агеев, И. Е. Кураленок, И. С. Некрестьянов // Российский семинар по оценке методов информационного поиска. Труды РОМИП'2010 (Казань, 15 октября 2010 г.). – Казань : Казанский государственный университет, 2010. – С. 172–187.

2. Большакова, Е. И. Автоматическая обработка текстов на естественном языке и компьютерная лингвистика : учеб. пособие / Е. И. Большакова, Э. С. Клышинский, Д. В. Ландэ [и др.]. – М. : МИЭМ, 2011. – 272 с.

3. Браславский, П. И. Интернет-поиск: принципы и практика : учеб. пособие / П. И. Браславский. – Екатеринбург : Изд-во Урал. ун-та, 2018. – 186 с.

4. Воронцов, К. В. Вероятностное тематическое моделирование: теория и практика / К. В. Воронцов. – М. : МЦНМО, 2020. – 742 с.

5. Гринева, М. П. Анализ текстовых документов для извлечения тематически сгруппированных ключевых терминов / М. П. Гринева, М. Н. Гринев // Труды Института системного программирования РАН. – 2009. – Т. 16. – С. 155–174.

6. Турдаков, Д. Ю. Методы и программные средства анализа текстов на основе семантической близости / Д. Ю. Турдаков // Труды Института системного программирования РАН. – 2010. – Т. 19. – С. 193–214.

7. Bird, S. Natural Language Processing with Python / S. Bird, E. Klein, E. Loper. – Sebastopol : O'Reilly Media, 2009. – 504 с.

8. Deerwester, S. Indexing by Latent Semantic Analysis / S. Deerwester, S. T. Dumais, G. W. Furnas [и др.] // Journal of the American Society for Information Science. – 1990. – Vol. 41, № 6. – С. 391–407.

9. Documentation for Gensim: Topic Modelling for Humans [Электронный ресурс]. – Режим доступа: https://radimrehurek.com/gensim/ (дата обращения: 15.04.2025).

10. Goldberg, Y. Neural Network Methods for Natural Language Processing / Y. Goldberg. – San Rafael : Morgan \& Claypool Publishers, 2017. – 309 с.

11. Honnibal, M. spaCy 2: Natural Language Understanding with Bloom Embeddings, Convolutional Neural Networks and Incremental Parsing / M. Honnibal, I. Montani // Sentometrics Research. – 2017. – 28 с.

12. Jurafsky, D. Speech and Language Processing: An Introduction to Natural Language Processing, Computational Linguistics, and Speech Recognition / D. Jurafsky, J. H. Martin. – 3rd ed. – Stanford : Pearson, 2023. – 628 с.

13. Le, Q. Distributed Representations of Sentences and Documents / Q. Le, T. Mikolov // Proceedings of the 31st International Conference on Machine Learning (Beijing, China, 21–26 June 2014). – 2014. – Vol. 32. – С. 1188–1196.

14. Manning, C. D. Introduction to Information Retrieval / C. D. Manning, P. Raghavan, H. Schütze. – Cambridge : Cambridge University Press, 2008. – 482 с.

15. Mihalcea, R. TextRank: Bringing Order into Texts / R. Mihalcea, P. Tarau // Proceedings of the 2004 Conference on Empirical Methods in Natural Language Processing (Barcelona, Spain, 25–26 July 2004). – 2004. – С. 404–411.

16. Mikolov, T. Efficient Estimation of Word Representations in Vector Space / T. Mikolov, K. Chen, G. Corrado, J. Dean // Proceedings of Workshop at ICLR. – 2013. – С. 1–12.

17. Rehurek, R. Software Framework for Topic Modelling with Large Corpora / R. Rehurek, P. Sojka // Proceedings of the LREC 2010 Workshop on New Challenges for NLP Frameworks (Valletta, Malta, 22 May 2010). – 2010. – С. 45–50.

18. Robertson, S. The Probabilistic Relevance Framework: BM25 and Beyond / S. Robertson, H. Zaragoza // Foundations and Trends in Information Retrieval. – 2009. – Vol. 3, № 4. – С. 333–389.

\clearpage